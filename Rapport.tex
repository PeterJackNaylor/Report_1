\documentclass{article}
\usepackage[utf8]{inputenc}
\usepackage[francais]{babel}
\usepackage[T1]{fontenc}
\usepackage{graphicx}    
\usepackage{eurosym}
\usepackage{verbatim}      
\usepackage{amsmath, amsthm}                             
\usepackage{latexsym}                               
\usepackage{amssymb}
\usepackage{tabularx}
\usepackage{setspace}
\usepackage{listings}
\usepackage{geometry}
\usepackage{fancyhdr}
\usepackage{enumitem}
\usepackage{colortbl}
\usepackage[dvipsnames]{xcolor}
\usepackage{booktabs} 
\usepackage{moreverb}

\DeclareMathAlphabet{\mathonebb}{U}{bbold}{m}{n}
\newcommand{\one}{\ensuremath{\mathonebb{1}}}

\usepackage{color}
\usepackage{multirow}
%\usepackage{float}
\definecolor{gris25}{gray}{0.75}
\usepackage{colortbl}
\usepackage{fancyhdr}
\usepackage{amsmath,amsfonts,amssymb}
\usepackage{titlesec}
\usepackage{supertabular}
\usepackage{longtable}

\usepackage{caption}
\usepackage{subcaption}


\usepackage{listings}
\definecolor{dkgreen}{rgb}{0,0.4,0}
\definecolor{gray}{rgb}{0.5,0.5,0.5}
\definecolor{mauve}{rgb}{0.58,0,0.82}

\usepackage{algorithm2e}
\newcommand{\algorithmicrequire}{\textbf{Input:}}
\newcommand{\algorithmicensure}{\textbf{Output:}}
\newcommand{\sign}{\text{sign}}
\newcommand{\argmin}{\text{argmin}}
\newcommand{\Cut}{\text{Cut}}
\newcommand{\Vol}{\text{Vol}}
\newcommand{\RCut}{\text{RCut}}
\newcommand{\NCut}{\text{NCut}}
\newcommand{\NCC}{\text{NCC}}
\newcommand{\RCC}{\text{RCC}}



\definecolor{dkyellow}{cmyk}{0, 0, 0.2, 0}
\lstset{
  language=Python,                % the language of the code
  basicstyle= \footnotesize,      % the size of the fonts that are used for the code
  numbers=left,                   % where to put the line-numbers
  numberstyle=\tiny\color{gray},  % the style that is used for the line-numbers
  stepnumber=2,                   % the step between two line-numbers. If it's 1, each line 
                                  % will be numbered
  showspaces=false,               % show spaces adding particular underscores
  showtabs=false,                 % show tabs within strings adding particular underscores
  frame=single,                   % adds a frame around the code
  rulecolor=\color{black},        % if not set, the frame-color may be changed on line-breaks within not-black text (e.g. commens (green here))
  tabsize=2,                      % sets default tabsize to 2 spaces
  captionpos=b,                   % sets the caption-position to bottom
  breaklines=true,                % sets automatic line breaking
  breakatwhitespace=false,        % sets if automatic breaks should only happen at whitespace
  keywordstyle=\color{blue},      % keyword style
  commentstyle=\color{dkgreen},   % comment style
  stringstyle=\color{mauve},       % string literal style
  backgroundcolor=\color{white},      % choose the background color. You must add \usepackage{color}
}

\usepackage{array}
\newcolumntype{L}[1]{>{\raggedright\let\newline\\\arraybackslash\hspace{0pt}}m{#1}}
\newcolumntype{C}[1]{>{\centering\let\newline\\\arraybackslash\hspace{0pt}}m{#1}}
\newcolumntype{R}[1]{>{\raggedleft\let\newline\\\arraybackslash\hspace{0pt}}m{#1}}

\usepackage{xcoffins}
\NewCoffin\tablecoffin
\NewDocumentCommand\Vcentre{m}
  {%
    \SetHorizontalCoffin\tablecoffin{#1}%
    \TypesetCoffin\tablecoffin[l,vc]%
  }



\begin{document}
%%%%%%%%%%%%%%%%%%%%%%%%%%%%%%%%%%%%%%%%%%%%%%%%%%%%%%%%%%%%%%%
\begin{titlepage}

%  \begin{center} 
%    \textsc{ENSAE ParisTech}\\
%    ---\\
%    Année 2013--2014
%  \end{center}
  
\begin{center}
\includegraphics[width=0.3\textwidth]{Mines_ParisTech.png} 
\includegraphics[width=0.3\textwidth]{CURIE.jpg}
\includegraphics[width=0.3\textwidth]{INSERM.jpg}
\end{center}

\vspace{\stretch{1}}
 
\noindent
\hrulefill
  \begin{center} \bfseries\Huge
Towards image-based cancer signatures from histopathology data
  \end{center}
  \begin{center} \huge
   First year report
  \end{center}
\hrulefill 
  
  \vspace{\stretch{2}}
   \begin{center}  \large
\textit{Supervisor} : \textsc{T. Walter}. \\
\textit{Units:} \textsc{Center for Computational Biology and UMR900}

   \end{center}
     
  \vspace{\stretch{3}}
  \begin{center} \Large
  Peter \textsc{Naylor}
  \end{center}

  \vspace{\stretch{4}}

  \begin{center}  \large
    June 2016
  \end{center}

\end{titlepage}
%%%%%%%%%%%%%%%%%%%%%%%%%%%%%%%%%%%%%%%%%%%%%%%%%%%%%%%%%%%%




\newpage
\tableofcontents
\newpage

\section{Introduction}


%\begin{figure}[!ht]
%\centering
%\includegraphics[width=\textwidth]{}
%\caption{}
%\label{}
%\end{figure}


\newpage

\begin{thebibliography}{9}

\bibitem{OriginalPaper} Thomas Bühler and Matthias Hein from Saarland University, Germany, \textit{Spectral Clustering based on the graph $p$-Laplacian}. ICML 2009.

\end{thebibliography}



\end{document}
